\chapter{Review of measure theory}

Most of the material in this chapter is from the book ``Measure Theory
and Fine Properties of Functions'' by Evans and Gariepy
\cite{evans1991measure}

\begin{defn}
  A \define{set function} $f$ is a mapping $f:P(x) \to Y$ where $Y$ is a
  Banach space. $Y$ is usually $\mathbb{R}$ or $\mathbb{C}$.
\end{defn}

\begin{defn}
  Let $X$ be a set. We let $P(X)$ denote the power set of $X$ i.e.\ the
  set of all subsets of $X$.
\end{defn}

\begin{defn}
  Let $X$ be a set. A subset $\mathfrak{M}\sub P(X)$ is called a
  $\sigma$-algebra on $X$ if $\mathfrak{M}$ has the following properties:
  \begin{enumerate}
    \item $\emptyset \in \mathfrak{M}$ and $X\in \mathfrak{M}$.
    \item $A\in \mathfrak{M}$ implies $A^c\in \mathfrak{M}$.
    \item $\mu\left( \sum_{i=1}^\infty A_i \right) =
      \sum_{i=1}^\infty\mu(A_i)$ whenever $\left\{ A_i
      \right\}_{i=1}^\infty$ is a countable colelction of disjoint sets
      in $ \mathfrak{M}$.
  \end{enumerate}
\end{defn}

\begin{defn}
  Suppose $X$ is a set and $\mathfrak{M}$ is a $\sigma$-algebra on
  $X$. A set function $\mu: \mathfrak{M} \to [0,\infty]$ is called a
  \define{measure} on $(X, \mathfrak{M})$ if $\mu$ has the following
  properties:
  \begin{enumerate}
    \item $\mu(\emptyset)=0$
    \item  $\mu(A)\le \sum_{i=1}^\infty \mu(A_k)$ whenever $A\sub
      \bigcup_{i=1}^\infty A_k$.
  \end{enumerate}
\end{defn}

A slightly more relaxed notion of a measure is what is called an outer measure.

\begin{defn}
  A set function $\gls{om}:P(X)\to [0,\infty]$ is called an \define{outer measure} if
  \begin{enumerate}[(i)]
    \item $\om(\emptyset) = 0$.
    \item Given a countable collection
      $\{A_k\}_{k=1}^\infty$ of subsets of $X$,
      \begin{equation*}
        \om\left(\bigcup_{k=1}^\infty A_k\right)\le \sum_{k=1}^\infty
        \om\left(A_k\right).
      \end{equation*}
  \end{enumerate}
\end{defn}

\begin{remark}
If $\om$ is an outer measure then we have that $\om(A)\le \om(B)$
whenever $A\sub B$ where $A,B\sub X$.
\end{remark}

\begin{defn}
  Given an outer measure $\om$ a subset $E\sub X$ is
  \define{$\om$-measurable} if it satisfies the following
  \define{Caratheadory Criterion}: 
  \begin{equation}
   \om(A) = \om(A\cap E)+\om(A\sm E), \for A\sub X.
  \end{equation}
  Equivalently we have that $E$ is $\om$-measurable iff
  $\om(A_1\cap A_2) = \om(A_1)+\om(A_2)$ whenever $\om(A_2\sub E)$ and
  $A_2\cap E=\eset$.
\end{defn}

\begin{prop}(exercise)
  Let $\om$ be an outer measure on $X$. If $\om(Z) = 0$ then $Z$ is
  $\om$-measurable.
\end{prop}

\begin{note}
  This says $\eset$ is measurable.  
\end{note}

\begin{prop}
  (exercise) Given an outer measure $\om$, the collection of all
  $\om$-measurable sets forms a $\sigma$-algebra.
\end{prop}

\begin{prop}(exercise)
  Any outer measure $\om$ restricted to its measurable sets is a
  measure.
\end{prop}

\begin{thm}(exercise)
  Suppose $\om$ is an outer measure and that $\left\{ A_k
  \right\}_{k=1}^\infty$ is a collection of $\om$-measurable sets.
  \begin{enumerate}
    \item If $\left\{ A_k\right\}_{k=1}^\infty$ are pairwise disjoint then
      $\mu\left( \cup_{k=1}^\infty A_k \right) =
      \sum_{k=1}^\infty\om(A_k)$.
    \item If $A_1\sub A_2 \sub \ldots$ then
      $\lim_{k\to \infty}\om(A_k) = \om\left( \cup_{k=1}^\infty A_k
      \right)$.
    \item If $A_1 \supseteq A_2 \supseteq \ldots$ and $\om(A_1) <
      \infty$ then $\lim_{k\to \infty}\om(A_k) = \om(\cap_{k=1}^\infty
      A_k)$.
  \end{enumerate}
\end{thm}

\begin{defn}
  Suppose $X$ is a topological space. The \define{Borel
  $\sigma$-algebra}, denoted $\mathfrak{B}_X$, is the smallest $\sigma$-algebra that contains all
  the open set. This can be constructed by taking the intersection of
  all $\sigma$-algebras on $X$ which contain the all open sets of
  $X$.
\end{defn}

\begin{defn}
  A subset $A\sub X$ is called $\sigma$-finite w.r.t.\ $\om$ if
  $A=\bigcup_{k=1}^\infty B_k$ where $B_k$ is a $\om$-measurable set and
  $\om(B_k)<\infty, \forall k$.
\end{defn}

\begin{defn}\mbox{}
  Let $X$ be an arbitrary space and $\om$ an outer measure on $X$.
  \begin{enumerate}
    \item $\om$ is \define{regular} if $\forall A\sub X$ there exists a
      $\om$-measurable set $B$ so that $A\sub B$ and $\om(A) = \om(B)$.
    \item $\om$ is a \define{Borel} measure if every Borel set is
      $\om$-measurable.
    \item $\om$ is \define{Borel regular} if $\om$ is Borel and
      $\forall A\sub \mathbb{R}$ (or $X$) there exists a Borel set
      $B$ with $A\sub B$ and $\om(A)=\om(B)$.
    \item $\om$ is a \define{Radon} measure if $\om$ is Borel and if
      $\om(K)<\infty$ for every compact set $K$.
  \end{enumerate}
\end{defn}

\begin{thm}\cite[thm 2]{evans1991measure}
  Let $\om$ be a regular outer measure on $X$. If $A_1\sub A_2 \sub \ldots$
  then $\lim_{k\to \infty}\om(A_k) = \om\left( \cup_{k=1}^\infty A_k.
  \right)$
\end{thm}

\begin{remark}
The significance of this theorem is that we are not requiring the $A_k$
to be $\om$-measurable.
\end{remark}

\begin{thm}\cite[thm 3]{evans1991measure}
  Suppose $\om$ is a Borel regular outer measure on $\mathbb{R}^n$ and $A\sub
  \mathbb{R}^n$ is $\om$-measurable with $\om(A)<\infty$. Then
  $\om\rvert_A$ is a Radon measure. Here $\om\rvert_A$ is defined as
  $\om\rvert_A(B)= \om(A\cap B)$ for ever $B\sub X$.
\end{thm}

\begin{lemma}\cite[lemma 1]{evans1991measure}
  Let $\om$ be a Borel regular outer measure on $ \mathbb{R}^n$. and let
  $B\in \mathfrak{B}_X$.
  \begin{enumerate}
    \item If $\om < \infty$ then for every $\ep>0$ there exists a closed
      set $C_\ep$ so that $C_\ep \sub B$ and $\om(B\sm C_\ep)<\ep$.
    \item If $\om$ is a Radon outer measure, then for ever $\ep>0$ there
      exists an open set $U_\ep$ so that $B\sub U_\ep$ and
    $\om(B\sm U_\ep)< \ep$.
  \end{enumerate}
\end{lemma}

\begin{thm}\cite[thm 4]{evans1991measure}(Approximation by open and
  compact sets)\\
  Let $\om$ be a Radon outer measure on $\mathbb{R}^n$. Then
  \begin{enumerate}
    \item for each set $A\sub \mathbb{R}^n$
      \begin{equation*}
        \om(A) = \inf\left\{ \om(U):A\sub U, U \text{ is open} \right\},
      \end{equation*}
      and
    \item for each $\om$-measurable set $A\sub \mathbb{R}^n$
      \begin{equation*}
        \om(A) = \sup\left\{ \om(K):K\sub A, K \text{ is compact}
      \right\}.
      \end{equation*}
  \end{enumerate}
\end{thm}



