\chapter{Review of measure theory}

Most of the material in this chapter is from the book ``Measure Theory and Fine Properties of
Functions''  by Evans and Gariepy \cite{evans1991measure}

\section{Monday, August 25, 2014}

\begin{defn}
  A \define{set function} $f$ is a mapping $f:P(x) \to Y$ where $Y$ is a
  Banach space. $Y$ is usually $\mathbb{R}$ or $\mathbb{C}$.
\end{defn}

\begin{defn}
  A set function \gls{om} is an \define{outer measure} if
  \begin{enumerate}[(i)]
    \item $\om(A)\ge 0,\, \forall A\subseteq X, \om(\emptyset)=0$
    \item (monotonicity) If $A\sub B$ then $\om(A)\le \om(B)$.
    \item (Subaditivity) Given a countable collection
      $\{A_k\}_{k=1}^\infty$ of subsets of $X$,
      \begin{equation*}
        \om\left(\bigcup_{k=1}^\infty A_k\right)\le \sum_{k=1}^\infty \om\left(A_k\right)
      \end{equation*}
  \end{enumerate}
\end{defn}

\begin{fact}
  Every outer measure induces a measure in the following way.
\end{fact}

\begin{defn}
  Given an outer measure $\om$ a subset $E\sub X$ is \define{$\om$-measurable} if
  it satisfies the following \define{Caratheadory Criterion}: $\forall
  A\sub X$ $\om(A) = \om(A\cap E)+\om(A\sm E)$. Notice $A$ does not have
  to necessarily be $\om$-measurable.

  Equivalently we have that $E$ is $\om$-measurable iff
  $\om(A_1\cap A_2) = \om(A_1)+\om(A_2)$ whenever $\om(A_2\sub E)$ and
  $A_2\cap E=\eset$.
\end{defn}

\begin{prop}(exercise)
  Let $\om$ be an outer measure on $X$. If $\om(Z) = 0$ then $Z$ is
  $\om$-measurable.
\end{prop}
\begin{note}
  This says $\eset$ is measurable. This also says the $E^c$ is
  measurable.
\end{note}
\begin{prop}
  (exercise) Given an outer measure $\om$ the collection of all
  $\om$-measurable sets forms a $\sigma$-algebra.
\end{prop}

\begin{defn}
  A set function $\gls{meas}$ is called a \define{measure} if it is an outer measure $\om$ restricted to its
  measurable sets.
\end{defn}

\begin{thm}(exercise)
  Suppose $\om$ is an outer measure and that $\left\{ A_k
  \right\}_{k=1}^\infty$ is a collection of $\om$-measurable sets.
  \begin{enumerate}
    \item If $\left\{ A_k\right\}_{k=1}^\infty$ are pairwise disjoint then
      $\mu\left( \cup_{k=1}^\infty A_k \right) =
      \sum_{k=1}^\infty\om(A_k)$.
    \item If $A_1\sub A_2 \sub \ldots$ then
      $\lim_{k\to \infty}\om(A_k) = \om\left( \cup_{k=1}^\infty A_k
      \right)$.
    \item If $A_1 \supseteq A_2 \supseteq \ldots$ and $\om(A_1) < \infty$ then
      $\lim_{k\to \infty}\om(A_k) = \om(\cap_{k=1}^\infty A_k)$.
  \end{enumerate}
\end{thm}

\begin{defn}
  Suppose $X$ is a topological space. The \define{Borel
  $\sigma$-algebra} is the smallest $\sigma$-algebra that contains all the
  open set.
\end{defn}

\begin{defn}
  A subest $A\sub X$ is $\sigma$-finite w.r.t.\ $\om$ if
  $A=\bigcup_{k=1}^\infty B_k$ where $B_k$ is a $\om$-measurable set and
  $\om(B_k)<\infty, \forall k$.
\end{defn}

\begin{defn}\mbox{}
  \begin{enumerate}
    \item $\om$ is \define{regular} if $\forall A\sub X$ there exists a
      $\om$-measurable set $B$ so that $A\sub B$ and $\om(A) = \om(B)$.
    \item $\om$ is a \define{Borel} measure if every Borel set is
      $\om$-measurable.
    \item $\om$ is \define{Borel regular} if $\om$ is Borel and
      $\forall A\sub \mathbb{R}$ (or $X$) there exists a Borel set
      $B$ with $A\sub B$ and $\om(A)=\om(B)$.
    \item $\om$ is a \define{Radon} measure if $\om$ is Borel and if
      $\om(K)<\infty$ for every compact set $K$.
  \end{enumerate}
\end{defn}

\begin{thm}
  Let $\om$ be a regular measure on $X$. If $A_1\sub A_2 \sub \ldots$
  then $\lim_{k\to \infty}\om(A_k) = \om\left( \cup_{k=1}^\infty A_k.
  \right)$
\end{thm}

\begin{remark}
The significance of this theorem is that we are not requiring the $A_k$
to be $\om$-measurable.
\end{remark}

\begin{thm}
  Suppose $\om$ is a Borel regular on $\mathbb{R}^n$ and $A\sub
  \mathbb{R}^n$ is $\om$-measurable with $\om(A)<\infty$. Then
  $\om\rvert_A$ is a Radon measure. Here $\om\rvert_A$ is defined as
  $\om\rvert_A(B)= \om(A\cap B)$.
\end{thm}

