\chapter{Review of measure theory}

\section{Outer Measures}
\label{sec:measure_theory}

Most of the material in this chapter is from the book ``Measure Theory
and Fine Properties of Functions'' by Evans and Gariepy
\cite{evans1991measure}

\begin{defn}
  A \define{set function} $f$ is a mapping $f:P(x) \to Y$ where $Y$ is a
  Banach space. $Y$ is usually $\mathbb{R}$ or $\mathbb{C}$.
\end{defn}

\begin{defn}
  Let $X$ be a set. We let $P(X)$ denote the power set of $X$ i.e.\ the
  set of all subsets of $X$.
\end{defn}

\begin{defn}
  Let $X$ be set. A non-empty subset $\mathfrak{M}\sub P(X)$ is called an
  \define{algebra} on $X$ if $\mathfrak{M}$ has the following properties:
  \begin{enumerate}
    \item $A\in \mathfrak{M}$ implies $A^c\in \mathfrak{M}$.
    \item If $n\in \N$ and $A_1,\ldots,A_n\in \mathfrak{M}$ then $
      \bigcup_{i=1}^\infty A_i\in \mathfrak{M}$
  \end{enumerate}
  A \define{$\sigma$-algebra} is an algebra which is closed under countable
  unions instead of finite unions.
\end{defn}

\begin{remark}
  Notice that if $\mathfrak{M}$ is an algebra or a $\sigma$-algebra on
  $X$ then $\emptyset\in \mathfrak{M}$ and $X\in \mathfrak{M}$ because
  if $E\in \mathfrak{M}$ then $\emptyset = E\cap E^{c}$ and $X = E\cup
  E^c$.
\end{remark}

\begin{defn}
  Suppose $X$ is a set and $\mathfrak{M}$ is a $\sigma$-algebra on
  $X$. A set function $\mu: \mathfrak{M} \to [0,\infty]$ is called a
  \define{measure} on $(X, \mathfrak{M})$ if $\mu$ has the following
  properties:
  \begin{enumerate}
    \item $\mu(\emptyset)=0$
    \item $\mu\left( \sum_{i=1}^\infty A_i \right) =
      \sum_{i=1}^\infty\mu(A_i)$ whenever $\left\{ A_i
      \right\}_{i=1}^\infty$ is a countable colelction of disjoint sets
      in $ \mathfrak{M}$.
  \end{enumerate}
\end{defn}

A slightly more relaxed notion of a measure is what is called an outer measure.

\begin{defn}
  A set function $\gls{om}:P(X)\to [0,\infty]$ is called an \define{outer measure} if
  \begin{enumerate}
    \item $\om(\emptyset)=0$
    \item $\om(A) \le \om(B)$ if $A\sub B\sub X$,
    \item  $\om\left( \bigcup_{i=1}^\infty A_i \right)\le
      \sum_{i=1}^\infty \om(A_k)$ whenever $\left\{ A_i
      \right\}_{i=1}^\infty\sub P(X)$.
  \end{enumerate}
\end{defn}

\begin{defn}
  Given an outer measure $\om$ a subset $E\sub X$ is
  \define{$\om$-measurable} if it satisfies the following
  \define{Caratheadory Criterion}:
  \begin{equation}
   \om(A) = \om(A\cap E)+\om(A\sm E), \for A\sub X.
  \end{equation}
  Equivalently we have that $E$ is $\om$-measurable iff
  $\om(A_1\cap A_2) = \om(A_1)+\om(A_2)$ whenever $\om(A_2\sub E)$ and
  $A_2\cap E=\eset$.
\end{defn}

\begin{prop}(exercise)
  Let $\om$ be an outer measure on $X$. If $\om(Z) = 0$ then $Z$ is
  $\om$-measurable.
\end{prop}

\begin{note}
  This says $\eset$ is measurable.
\end{note}

\begin{prop}
  (exercise) Given an outer measure $\om$, the collection of all
  $\om$-measurable sets forms a $\sigma$-algebra.
\end{prop}

\begin{defn}
  Denote the $\sigma$-algebra of $\om$-mesaurable sets as
  $\mathfrak{M}_{\om}$.

\end{defn}

\begin{prop}(exercise)
  Any outer measure $\om$ restricted to its measurable sets is a
  measure.
\end{prop}

\begin{thm}(exercise)
  Suppose $\om$ is an outer measure and that $\left\{ A_k
  \right\}_{k=1}^\infty$ is a collection of $\om$-measurable sets.
  \begin{enumerate}
    \item If $\left\{ A_k\right\}_{k=1}^\infty$ are pairwise disjoint then
      $\mu\left( \cup_{k=1}^\infty A_k \right) =
      \sum_{k=1}^\infty\om(A_k)$.
    \item If $A_1\sub A_2 \sub \ldots$ then
      $\lim_{k\to \infty}\om(A_k) = \om\left( \cup_{k=1}^\infty A_k
      \right)$.
    \item If $A_1 \supseteq A_2 \supseteq \ldots$ and $\om(A_1) <
      \infty$ then $\lim_{k\to \infty}\om(A_k) = \om(\cap_{k=1}^\infty
      A_k)$.
  \end{enumerate}
\end{thm}

\begin{defn}
  Suppose $X$ is a topological space. The \define{Borel
  $\sigma$-algebra}, denoted $\mathfrak{B}_X$, is the smallest $\sigma$-algebra that contains all
  the open set. This can be constructed by taking the intersection of
  all $\sigma$-algebras on $X$ which contain the all open sets of
  $X$.
\end{defn}

\begin{defn}
  A subset $A\sub X$ is called $\sigma$-finite w.r.t.\ $\om$ if
  $A=\bigcup_{k=1}^\infty B_k$ where $\left\{ B_i \right\}_{i=1}^\infty\sub \mom$ and
  $\om(B_i)<\infty, \forall i$.
\end{defn}

\begin{defn}\mbox{}
  Let $X$ be an arbitrary space and $\om$ an outer measure on $X$.
  \begin{enumerate}
    \item $\om$ is \define{regular} if $\forall A\sub X$ there exists a
      $B\in\mom$ so that $A\sub B$ and $\om(A) = \om(B)$.
    \item $\om$ is a \define{Borel} measure if every Borel set is
      $\om$-measurable.
    \item $\om$ is \define{Borel regular} if $\om$ is Borel and
      $\forall A\sub \mathbb{R}$ (or $X$) there exists a Borel set
      $B$ with $A\sub B$ and $\om(A)=\om(B)$.
    \item $\om$ is a \define{Radon} measure if $\om$ is Borel regular and if
      $\om(K)<\infty$ for every compact set $K$.
  \end{enumerate}
\end{defn}

\begin{thm}\cite[sec 1.1]{evans1991measure}
  Let $\om$ be a regular outer measure on $X$. If $A_1\sub A_2 \sub \ldots$
  then $\lim_{k\to \infty}\om(A_k) = \om\left( \cup_{k=1}^\infty A_k.
  \right)$
\end{thm}

\begin{remark}
The significance of this theorem is that we are not requiring the $A_k$
to be $\om$-measurable.
\end{remark}

\begin{defn}
  Let $\om\rvert_A$ be defined as
  $\om\rvert_A(B)= \om(A\cap B)$ for ever $B\sub X$.
\end{defn}

\begin{thm}\cite[sec 1.1]{evans1991measure}
  Suppose $\om$ is a Borel regular outer measure on $\mathbb{R}^n$ and $A\sub
  \mathbb{R}^n$ is $\om$-measurable with $\om(A)<\infty$. Then
  $\om\rvert_A$ is a Radon measure.
\end{thm}

\begin{lemma}\cite[sec 1.1]{evans1991measure}
  Let $\om$ be a Borel regular outer measure on $ \mathbb{R}^n$. and let
  $B\in \mathfrak{B}_X$.
  \begin{enumerate}
    \item If $\om < \infty$ then for every $\ep>0$ there exists a closed
      set $C_\ep$ so that $C_\ep \sub B$ and $\om(B\sm C_\ep)<\ep$.
    \item If $\om$ is a Radon outer measure, then for ever $\ep>0$ there
      exists an open set $U_\ep$ so that $B\sub U_\ep$ and
    $\om(B\sm U_\ep)< \ep$.
  \end{enumerate}
\end{lemma}

\begin{thm}\cite[sec 1.1]{evans1991measure}(Approximation by open and
  compact sets)\\
  Let $\om$ be a Radon outer measure on $\mathbb{R}^n$. Then
  \begin{enumerate}
    \item for each set $A\sub \mathbb{R}^n$
      \begin{equation*}
        \om(A) = \inf\left\{ \om(U):A\sub U, U \text{ is open} \right\},
      \end{equation*}
      and
    \item for each $\om$-measurable set $A\sub \mathbb{R}^n$
      \begin{equation*}
        \om(A) = \sup\left\{ \om(K):K\sub A, K \text{ is compact}
      \right\}.
      \end{equation*}
  \end{enumerate}
\end{thm}

\begin{thm}\cite[sec 1.1]{evans1991measure}(Caratheodory's Criterion)
  Let $\om$ be an outer measure on $ \mathbb{R}^n$. If $\om(A\cup B) =
  \om(A)+\om(B)$ for all sets $A,B\sub \mathbb{R}^n$ with
  $\text{dist}(A,B)>0$, then $\om$ is a Borel outer measure.
\end{thm}

% end section 'sec:measure_theory'

\section{Covering Lemmas}
\label{sec:covering_lemmas}

\begin{thm}[Vitali's covering lemma]\cite[sec 1.5.1]{evans1991measure}
Let $\mcal{F}$ be an collection of nondegenerate closed balls in
$\mathbb{R}^n$ with $\sup\left\{ \text{diam}(B) | B\in \mcal{F} \right\}
< \infty$.
  Then there exists a countable family $\mcal{G}$ of disjoint balls
  in $\mcal{F}$ such that
  \begin{equation*}
    \bigcup_{B\in \mcal{F}}B \sub \bigcup_{B\in
      \mcal{G}}5B.
  \end{equation*}
\end{thm}

If $\om$ is an arbitrary Radon outer measure on
$\mathbb{R}^n$, we can't always control $\om(5B)$ in terms of
$\om(B)$. So instead of enlarging the balls to cover the space we
try to control the number of overlaps by a constant which depends
only on the dimension. This is the content of the next theorem.

\begin{thm}(Besicovitch's covering theorem
    )\cite[sec 1.5.2]{evans1991measure}
  There exists a constant $M=M_n\in \mathbb{N}$ depending only on $n$, with the
  following property: If $\mcal{F}$ is any collection of
  nondegenerate closed balls in $\mathbb{R}^n$ with
  \begin{equation*}
    \sup\left\{ \text{diam}(B) | B\in \mcal{F} \right\}
< \infty
  \end{equation*}
  and if $\mcal{A}$ is the set of centers of balls in $\mcal{F}$,
  then there exist $\mcal{G}_1,\ldots,\mcal{G}_{M} \sub \mcal{F}$
  such that $\mcal{G}_i$ ($i=1,\ldots,M)$ is a countable collection
  of disjoint balls in $\mcal{F}$ and
  \begin{equation*}
    \mcal{A} \sub \bigcup_{i=1}^M\bigcup_{B\in \mcal{G}_i}B.
  \end{equation*}
\end{thm}

The following decompostion/covering theorem says you can cover an
arbitrary nonempty open set $\mathbb{R}^n$ with closed cubes whose side
lengths are proportional to the cube's distance to the boundaries
of the set. Also the number of neighboring cubes of any given cube is bounded according
to the cube's side length.

\begin{thm}(Whitney
  decomposition)\cite[appendix J]{grafakos2008classical}
  Let $\Omega$ be an open, nonempty, proper subset of
  $\mathbb{R}^n$. Then there exists family of closed cubes $\left\{
  Q_j \right\}_{j}$ such that
  \begin{enumerate}
    \item $\bigcup_j Q_j = \Omega$ and the $Q_j$'s have disjoint
      interiors.
    \item $\sqrt{n}\ell(Q_j)\le
      \text{dist}(Q_j,\Omega^c) \le 4\sqrt{n}\ell(Q_j)$.
    \item If the boundaries of two cubes $Q_j$ and $Q_k$ touch then
      \begin{equation*}
        \frac{1}{4} \le \frac{\ell(Q_j)}{\ell(Q_k)} \le 4.
      \end{equation*}
    \item For a given $Q_j$ there exists at most $12^n$ $Q_k$'s
      that touch it.
  \end{enumerate}

\end{thm}

% end section 'sec:covering_lemmas'

\section{Hausdorff Measure}
\label{sec:hausdorff_measure}

This section follows section of 2.1 of
\cite{evans1991measures}.

\begin{defn}
  \begin{enumerate}\mbox{}
    \item Let $A\sub \mathbb{R}^n$, $0\le s < \infty$, $\delta \le \infty$.
    Define
    \begin{equation*}
      \mcal{H}_\delta^s \deq \inf\left\{ \sum_{j=1}^\infty
      \alpha(s)\left( \frac{\text{diam}(C_j)}{2} \right)^s : A\sub
      \bigcup_{j=1}^\infty C_j, \text{diam}(C_j) \le \delta  \right\}
    \end{equation*}
    where
    \begin{equation*}
      \alpha(s) \deq \frac{\pi^{s/2}}{\Gamma\left(\frac{s}{2}+1\right)}.
    \end{equation*}
    Here $\Gamma$ is the Gamma function of complex analysis.
    \item For $A$ and $s$ as above, define
      \begin{equation*}
        \mcal{H}^s \deq \lim_{\delta\to 0}\mcal{H}_{\delta}^{s}(A) =
        \sup_{\delta > 0}\mcal{H}_{\delta}^s(A).
      \end{equation*}
      We call $\mcal{H}^s$ the \define{$s$-dimensional Hausdorff measure}.
  \end{enumerate}
\end{defn}
\begin{remark}\mbox{}
  \begin{enumerate}
    \item Notice that $\mcal{L}^n\left( B(x,r) \right) =
      \alpha(n)r^n$
      for all balls $B(x,r)\sub \mathbb{R}^n$. We will see later that
      if $s=k$ is an integer $\mcal{H}^k$ agrees with the ordinary
      ``$k$-dimensional surface area'' on nice sets. This is the reason
      we include the normalizing constant $\alpha(s)$, in the
      definition.
  \end{enumerate}
\end{remark}


% end section 'sec:hausdorff_measure'
