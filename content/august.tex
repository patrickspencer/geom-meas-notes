\section{Monday, August 25, 2014}

\subsection{Review of measure theory}

\begin{defn}
  A \define{set function} is a mapping $\mu:P(x) \to Y$ where $Y$ is a
  Banach space. $Y$ is usually $\mathbb{R}$ or $\mathbb{C}$.
\end{defn}

\begin{defn}
  A set function is an \define{outer measure} if
  \begin{enumerate}[(i)]
    \item $\mu(A)\ge 0,\, \forall A\subseteq X, \mu(\emptyset)=0$
    \item (monotonicity) If $A\sub B$ then $\mu(A)\le \mu(B)$.
    \item (Subaditivity) Given a countable collection
      $\{A_k\}_{k=1}^\infty$ of subsets of $X$,
      \begin{equation*}
        \mu(\cup_{k=1}^\infty A_k)\le \mu\left(\sum_{k=1}^\infty A_k\right)
      \end{equation*}
  \end{enumerate}
\end{defn}

\begin{fact}
  Every outer measure includes a measure in the following way.
\end{fact}

\begin{defn}
  Given an outmeasure $\mu$ a subset $E\sub X$ is \define{measurable} if
  it satisfies the following \define{Caratheadory Criterion}: $\forall
  A\sub X$ $\mu(A) = \mu(A\cap E)+\mu(A\sm E)$. Notice $A$ does not have
  to necessarily be $\mu$-measurable.

  Equivalently we have that $E$ is $\mu$-measurable iff
  $\mu(A_1\cap A_2) = \mu(A_1)+\mu(A_2)$ whenever $\mu(A_2\sub E)$ and
  $A_2\cap E=\eset$.
\end{defn}

\begin{prop}(exercise)
  Let $\mu$ be an outer measure on $X$. If $\mu(Z) = 0$ then $Z$ is
  $\mu$-measurable.
\end{prop}
\begin{note}
  This says $\eset$ is measurable. This also says the $E^c$ is
  measurable.
\end{note}
\begin{prop}
  (exercise) Given an outer measure $\mu$ the collection of all
  $\mu$-measurable sets forms a $\sigma$-algebra.
\end{prop}

\begin{defn}
  An outer measure $\mu$ is a \define{mesaure} when restricted to its
  measurable sets.
\end{defn}

\begin{thm}(exercise)
  Suppose $\mu$ is an outer measure and that $\left\{ A_k
  \right\}_{k=1}^\infty$ is a collection of $\mu$-measurable sets.
  \begin{enumerate}
    \item If $\left\{ A_k\right\}_{k=1}^\infty$ are pairwise disjoint then
      $\sigma\left( \cup_{k=1}^\infty A_k \right) =
      \sum_{k=1}^\infty\mu(A_k)$.
    \item If $A_1 \sub \ldots A_k\sub A_{k+1} \sub \ldots$ then
      $\lim_{k\to \infty}\mu(A_k) = \mu\left( \cup_{k=1}^\infty A_k
      \right)$.
    \item If $A_1 \supseteq A_2 \supseteq \ldots$ and $\mu(A_1) < \infty$ then
      $\lim_{k\to \infty}\mu(A_k) = \mu(\cap_{k=1}^\infty A_k)$.
  \end{enumerate}
\end{thm}

\begin{defn}
  Suppose $X$ is a topological space. The \define{Borel
  $\sigma$-algebra} is the smallest $\sigma$-alg. that contains all the
  open set.
\end{defn}

\begin{defn}
  A subest $A\sub X$ is $\sigma$-finite w.r.t.\ $\mu$ if
  $A=\Cup_{k=1}^\infty B_k$ where $B_k$ is a $\mu$-measurable set and
  $\mu(B_k)<\infty, \forall k$.
\end{defn}

\begin{defn}
  \begin{enumerate}
    \item $\mu$ is \define{regular} if \forall $A\sub X$ there exists a
      $\mu$-measurable set $B$ so that $A\sub B$ and $\mu(A) = \mu(B)$.
    \item $\mu$ is a \define{Borel} measure if every Borel set is
      $\mu$-measurable.
    \item $\mu$ is \define{Borel regular} if $\mu$ is Borel and
      $\forall A\sub \mathbb{R}$ (or $X$) there exists a Borel set
      $B$ with $A\sub B$ and $\mu(A)=\mu(B)$.
    \item $\mu$ is a \define{Radon} measure if $\mu$ is Borel and if
      $\mu(K)<\infty$ for every cpt set $K$.
  \end{enumerate}
\end{defn}

\begin{thm}
  Let $\mu$ be a regular measure on $X$. If $A_1\sub A_2 \sub \ldots$
  then $\lim_{k\to \infty}\mu(A_k) = \mu\left( \cup_{k=1}^\infty A_k.
  \right)$
\end{thm}

the significance of this theorem is that we are not requiring the $A_k$
to be $\mu$-measurable.

\begin{thm}
  Suppose $\mu$ is a Borel regular on $\mathbb{R}^n$ and $A\sub
  \mathbb{R}^n$ is $\mu$-measurable with $\mu(A)<\infty$. Then
  $\mu\rvert_A$ is a Radon measure. Here $\mu\rvert_A$ is defined as
  $\mu\rvert_A(B)= \mu(A\cap B)$.
\end{thm}

