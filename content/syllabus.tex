\chapter{Intro and Syllabus}

The following is basic information about the class from the syllabus.

\textbf{Institution}: University of Missouri at Columbia

\textbf{Semester}: Fall 2014

\textbf{Instructor}: Steve Hofmann

\textbf{Course title}: Math 8302 - Introduction to Geometric Measure Theory and
Quantitative Rectifiablity.

\textbf{Description}: Geometric Measure Theory (GMT) is largely concerned with
the notion of ``rectifiability'', which describes a sense in which a
general set of points (in $\mathbb{R}^n$, say) is approximated by
``nice'' surfaces. In the past couple of decades, this concept has been
sharpened: ``uniform (i.e., quantitative) rectifiability'' entails
approximation of a set by nice surfaces, but in a quanitatively precise
way, which turns out to have deep connections with the behavior, on the
set, of singular integral operators, square functions, and harmonic
measure. We plan to cover the following topics, to the extent that time
permits
\begin{enumerate}
  \item Review of Hausdorff measure
  \item Bounded variation, sets of finite perimeter,
    isoperimetric inequality, reduced boundary, measure theoretic
    boundary, Hauss-Green theorem.
  \item Ahlfors-David Regularity, ``Spaces of homogeneous type'' and
    M.\ Christ's dyadic cube reconstruction.
  \item NTA domains and ``big pieces of Lipschitz graphs''.
  \item Uniform rectifiablity, Corona decomposition, ``Geometric
    Lemma'' and the ``Bilateral Weak Geometric Lemma'', singular
    intergrals and square functions.
\end{enumerate}

\textbf{Prerequisites} Familiarity with the elemetntary theory of measure and
  integration and with the basic subject matter of harmonic analysis:
  Fourier transform, Hardy-Littlewood maximal function, approximate
  identities, Whitney decomposition, Littlewood-Paley theory,
  classical Calderon-Zygmund theory, BMO, Carleson measures, theory of
  sungular intergrals on Lipschitz graphs.

  \textbf{Text}: There is no offical textbook for the course, however,
  much of the material presented in the course will follow
  \cite{david1991singular},\cite{david1993analysis},
  \cite{evans1991measure}, and \cite{wheeden1977measure}. Other material
  may be taken directly from the literature.

